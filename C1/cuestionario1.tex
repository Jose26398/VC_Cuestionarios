\documentclass[11pt,a4paper]{article}
\usepackage[spanish,es-nodecimaldot]{babel}	% Utilizar español
\usepackage[utf8]{inputenc}					% Caracteres UTF-8
\usepackage{graphicx}						% Imagenes
\usepackage[hidelinks]{hyperref}			% Poner enlaces sin marcarlos en rojo
\usepackage{fancyhdr}						% Modificar encabezados y pies de pagina
\usepackage{float}							% Insertar figuras
\usepackage[textwidth=390pt]{geometry}		% Anchura de la pagina
\usepackage[nottoc]{tocbibind}				% Referencias (no incluir num pagina indice en Indice)
\usepackage{enumitem}						% Permitir enumerate con distintos simbolos
\usepackage[T1]{fontenc}					% Usar textsc en sections
\usepackage{amsmath}						% Símbolos matemáticos
\usepackage{listings}
\usepackage{color}

 
\definecolor{codegreen}{rgb}{0,0.6,0}
\definecolor{codegray}{rgb}{0.5,0.5,0.5}
\definecolor{codepurple}{rgb}{0.58,0,0.82}
\definecolor{backcolour}{rgb}{0.99,0.99,0.99}
 
\lstdefinestyle{mystyle}{
    backgroundcolor=\color{backcolour},   
    commentstyle=\color{codegreen},
    keywordstyle=\color{magenta},
    numberstyle=\tiny\color{codegray},
    stringstyle=\color{codepurple},
    basicstyle=\footnotesize,
    breakatwhitespace=false,         
    breaklines=true,                 
    captionpos=b,                    
    keepspaces=true,                 
    numbers=left,                    
    numbersep=5pt,                  
    showspaces=false,                
    showstringspaces=false,
    showtabs=false,                  
    tabsize=2
}
 
\lstset{style=mystyle, language=Python}

% Comando para poner el nombre de la asignatura
\newcommand{\asignatura}{Visión por Computador}
\newcommand{\autor}{José María Sánchez Guerrero}
\newcommand{\titulo}{Cuestionario 1}
\newcommand{\subtitulo}{Filtrado y Detección de regiones}

% Configuracion de encabezados y pies de pagina
\pagestyle{fancy}
\lhead{\autor{}}
\rhead{\asignatura{}}
\lfoot{Grado en Ingeniería Informática}
\cfoot{}
\rfoot{\thepage}
\renewcommand{\headrulewidth}{0.4pt}		% Linea cabeza de pagina
\renewcommand{\footrulewidth}{0.4pt}		% Linea pie de pagina

\begin{document}
\pagenumbering{gobble}

% Pagina de titulo
\begin{titlepage}

\begin{minipage}{\textwidth}

\centering

\includegraphics[scale=0.5]{img/ugr.png}\\

\textsc{\Large \asignatura{}\\[0.2cm]}
\textsc{GRADO EN INGENIERÍA INFORMÁTICA}\\[1cm]

\noindent\rule[-1ex]{\textwidth}{1pt}\\[1.5ex]
\textsc{{\Huge \titulo\\[0.5ex]}}
\textsc{{\Large \subtitulo\\}}
\noindent\rule[-1ex]{\textwidth}{2pt}\\[3.5ex]

\end{minipage}

\vspace{0.5cm}

\begin{minipage}{\textwidth}

\centering

\textbf{Autor}\\ {\autor{}}\\[2.5ex]
\textbf{Rama}\\ {Computación y Sistemas Inteligentes}\\[2.5ex]
\vspace{0.3cm}

\includegraphics[scale=0.3]{img/etsiit.jpeg}

\vspace{0.7cm}
\textsc{Escuela Técnica Superior de Ingenierías Informática y de Telecomunicación}\\
\vspace{1cm}
\textsc{Curso 2019-2020}
\end{minipage}
\end{titlepage}

\pagenumbering{arabic}
\tableofcontents
\thispagestyle{empty}				% No usar estilo en la pagina de indice

\newpage

\setlength{\parskip}{1em}


\section*{Ejercicio 1}
\addcontentsline{toc}{section}{Ejercicio 1}

\textbf{Diga en una sola frase cuál cree que es el objetivo principal de la Visión por Computador. Diga también cuál
es la principal propiedad de las imágenes de cara a la creación algoritmos que la procesen.}

El objetivo principal de la Visión por Computador es obtener información significativa de imágenes digitales, posteriormente
analizarla, tratarla y comprender su contenido para tomar decisiones sobre ella de la forma más similar posible a la humana.

Pese a la respresentación en forma matricial de las imágenes que facilita los cálculos a los algoritmos, no podemos centrarnos
únicamente en una posición (píxel) de ésta, ya que los valores alrededor suyo también contienen información relevante sobre él.


\section*{Ejercicio 2}
\addcontentsline{toc}{section}{Ejercicio 2}

\textbf{Expresar las diferencias y semejanzas entre las operaciones de correlación y convolución. Dar una interpretación
de cada una de ellas que en el contexto de uso en visión por computador.}

Ambas son operaciones que transforman localmente una imagen calculado nuevos valores para cada uno de los píxeles. Esto lo hacen
utilizando una máscara 2D de tamaño $NxN$, siendo $N$ un número impar, y aplicándola de la siguiente forma:

Para la correlación:
\begin{equation}
G[i,j]= \sum_{u=-k}^{k} \sum_{v=-k}^{k} H[u,v]F[i+u,j+v]
\end{equation}

Para la convolución:
\begin{equation}
G[i,j]= \sum_{u=-k}^{k} \sum_{v=-k}^{k} H[u,v]F[i-u,j-v]
\end{equation}

Como podemos ver, correlación y convolución son prácticamente iguales, excepto que en la convolución volteamos el filtro antes de
correlacionar. Por ejemplo, convolucionar una imagen 1D con un filtro $(1,3,5)$ sería lo mismo que correlacionarla con el filtro
$(5,3,1)$. En caso de que fuese una convolución 2D voltearíamos tanto horizontal como verticalmente.

Otra cosa que tienen en común es que, como ambos son filtros lineales, ambos cumplen las siguientes propiedades: la \textbf{
superposición}, la cual dice que es lo mismo aplicar una máscara a una composición de imágenes $f_1+f_2$, que aplicarsela a
$f_1$ y a $f_2$ por separado: $h*(f_1 + f_2) = (h*f_1) + (h*f_2)$; y son \textbf{\textit{Shift-Invariant System}}, es decir,
sistemas cuyo valor de entrada no cambian los valores de salida, por lo que no dependen de ellos.

La diferencias más importantes entre ellos es que la convolución es \textbf{conmutativa}, \textbf{distributiva en la adición} y, la
más relevante, \textbf{asociativa}. Es decir, siendo $f$ y $g$ dos filtros distintos, entonces $(f * g)*h = f*(g * h)$. Esto es muy
útil por ejemplo para calcular el filtro \textit{Difference of Gaussian (DoG)}, en el que no tendríamos que convolucionar la imagen
con un filtro Gaussiano y posteriormente con uno derivado, simplemente convolucionaríamos el filtro Gaussiano con el derivado y ya se
lo aplicamos a la imagen.


\section*{Ejercicio 3}
\addcontentsline{toc}{section}{Ejercicio 3}

\textbf{¿Cuál es la diferencia “esencial” entre el filtro de convolución y el de mediana? Justificar la respuesta.}

La principal diferencia entre estos filtros es que el filtro de convolución, como hemos visto antes, es lineal, mientras que el
filtro de mediana no lo es. Para justificar la respuesta veamos un ejemplo: si tenemos los filtros $A=(0,1,2,3,4)$, $B=(5,0,1,4,5)$
y $(A+B)=(5,1,3,7,9)$, el cálculo de sus medianas es:
\begin{equation}
Med(A) = 2 \hspace{10mm} Med(B) = 4
\end{equation}
Y por tanto, como:

\begin{equation}
Med(A)+Med(B)=6 \hspace{5mm} \neq \hspace{5mm} Med(A+B)=5
\end{equation}

podemos justificar que los filtros de mediana no son lineales.

\newpage


\section*{Ejercicio 4}
\addcontentsline{toc}{section}{Ejercicio 4}

\textbf{Identifique el “mecanismo concreto” que usa un filtro de máscara para transformar una imagen.}

Usan información local en vez de global

\end{document}