\documentclass[11pt,a4paper]{article}
\usepackage[spanish,es-nodecimaldot]{babel}	% Utilizar español
\usepackage[utf8]{inputenc}					% Caracteres UTF-8
\usepackage{graphicx}						% Imagenes
\usepackage[hidelinks]{hyperref}			% Poner enlaces sin marcarlos en rojo
\usepackage{fancyhdr}						% Modificar encabezados y pies de pagina
\usepackage{float}							% Insertar figuras
\usepackage[textwidth=390pt]{geometry}		% Anchura de la pagina
\usepackage[nottoc]{tocbibind}				% Referencias (no incluir num pagina indice en Indice)
\usepackage{enumitem}						% Permitir enumerate con distintos simbolos
\usepackage[T1]{fontenc}					% Usar textsc en sections
\usepackage{amsmath}						% Símbolos matemáticos
\usepackage{listings}
\usepackage{color}

 
\definecolor{codegreen}{rgb}{0,0.6,0}
\definecolor{codegray}{rgb}{0.5,0.5,0.5}
\definecolor{codepurple}{rgb}{0.58,0,0.82}
\definecolor{backcolour}{rgb}{0.99,0.99,0.99}
 
\lstdefinestyle{mystyle}{
    backgroundcolor=\color{backcolour},   
    commentstyle=\color{codegreen},
    keywordstyle=\color{magenta},
    numberstyle=\tiny\color{codegray},
    stringstyle=\color{codepurple},
    basicstyle=\footnotesize,
    breakatwhitespace=false,         
    breaklines=true,                 
    captionpos=b,                    
    keepspaces=true,                 
    numbers=left,                    
    numbersep=5pt,                  
    showspaces=false,                
    showstringspaces=false,
    showtabs=false,                  
    tabsize=2
}
 
\lstset{style=mystyle, language=Python}

% Comando para poner el nombre de la asignatura
\newcommand{\asignatura}{Visión por Computador}
\newcommand{\autor}{José María Sánchez Guerrero}
\newcommand{\titulo}{Cuestionario 2}
\newcommand{\subtitulo}{Clasificación de escenas y objetos}

% Configuracion de encabezados y pies de pagina
\pagestyle{fancy}
\lhead{\autor{}}
\rhead{\asignatura{}}
\lfoot{Grado en Ingeniería Informática}
\cfoot{}
\rfoot{\thepage}
\renewcommand{\headrulewidth}{0.4pt}		% Linea cabeza de pagina
\renewcommand{\footrulewidth}{0.4pt}		% Linea pie de pagina

\begin{document}
\pagenumbering{gobble}

% Pagina de titulo
\begin{titlepage}

\begin{minipage}{\textwidth}

\centering

\includegraphics[scale=0.5]{img/ugr.png}\\

\textsc{\Large \asignatura{}\\[0.2cm]}
\textsc{GRADO EN INGENIERÍA INFORMÁTICA}\\[1cm]

\noindent\rule[-1ex]{\textwidth}{1pt}\\[1.5ex]
\textsc{{\Huge \titulo\\[0.5ex]}}
\textsc{{\Large \subtitulo\\}}
\noindent\rule[-1ex]{\textwidth}{2pt}\\[3.5ex]

\end{minipage}

\vspace{0.5cm}

\begin{minipage}{\textwidth}

\centering

\textbf{Autor}\\ {\autor{}}\\[2.5ex]
\textbf{Rama}\\ {Computación y Sistemas Inteligentes}\\[2.5ex]
\vspace{0.3cm}

\includegraphics[scale=0.3]{img/etsiit.jpeg}

\vspace{0.7cm}
\textsc{Escuela Técnica Superior de Ingenierías Informática y de Telecomunicación}\\
\vspace{1cm}
\textsc{Curso 2019-2020}
\end{minipage}
\end{titlepage}

\pagenumbering{arabic}
\tableofcontents
\thispagestyle{empty}				% No usar estilo en la pagina de indice

\newpage

\setlength{\parskip}{1em}


\section*{Ejercicio 1}
\addcontentsline{toc}{section}{Ejercicio 1}
\textbf{¿Cuál es la transformación más fuerte de la geometría de una escena que puede introducirse al tomar una foto de ella? Dar algún ejemplo.}

\section*{Ejercicio 2}
\addcontentsline{toc}{section}{Ejercicio 2}
\textbf{Por qué es necesario usar el plano proyectivo para estudiar las transformaciones en las imágenes de fotos de escenas? Dar algún ejemplo.}

\section*{Ejercicio 3}
\addcontentsline{toc}{section}{Ejercicio 3}
\textbf{Sabemos que en el plano proyectivo un punto no existe en el sentido del plano afín, sino que se define por una clase de equivalencia de
vectores definida por . Razone usando las coordenadas proyectivas de los puntos afines de una recta que pase por el (0,0) del plano
afín y verifique que los punto de la recta del infinito del plano proyectivo son necsariamente vectores del tipo (*,*,0) con *=cualquier número.}


\section*{Ejercicio 4}
\addcontentsline{toc}{section}{Ejercicio 4}
\textbf{¿Qué propiedades de la geometría de un plano quedan invariantes cuando se toma una foto de él? Justificar la respuesta.}


\section*{Ejercicio 5}
\addcontentsline{toc}{section}{Ejercicio 5}
\textbf{En coordenadas homogéneas los puntos y rectas del plano se representan por vectores de tres coordenadas (notados x y l respectivamente), de
manera que si una recta contiene a un punto se verifica la ecuación 
. Considere una homografía H que transforma vectores de puntos, x= Hx. Dado que una homografía transforma vectores de tres coordenadas también
existen homografías G para transformar vectores de rectas l= Gl. Suponga una recta l y un punto x que verifican xTl=0 en el plano proyectivo y
suponga que conoce una homografía H que transforma vectores de puntos. En estas condiciones ¿cuál es la homografía G que transforma los vectores de
las rectas? Deducirla matemáticamente.}

\section*{Ejercicio 6}
\addcontentsline{toc}{section}{Ejercicio 6}
\textbf{¿Cuál es el mínimo número de escalares necesarios para fijar una homografía general? ¿Y si la homografía es afín? Justificar la respuesta}


\section*{Ejercicio 7}
\addcontentsline{toc}{section}{Ejercicio 7}
\textbf{Defina una homografía entre planos proyectivos que haga que el punto (3,0,2) del plano proyectivo-1 se transforme en un punto de la recta del
infinito del plano proyectivo-2? Justificar la respuesta.}


\section*{Ejercicio 8}
\addcontentsline{toc}{section}{Ejercicio 8}
\textbf{}

\section*{Ejercicio 9}
\addcontentsline{toc}{section}{Ejercicio 9}
\textbf{¿Cuáles son las propiedades necesarias y suficientes para que una matriz defina un movimiento geométrico no degenerado entre planos? Justificar
la respuesta}


\section*{Ejercicio 10}
\addcontentsline{toc}{section}{Ejercicio 10}
\textbf{¿Qué información de la imagen usa el detector de Harris para seleccionar puntos? ¿El detector de Harris detecta patrones geométricos o fotométricos?
Justificar la contestación.}


\section*{Ejercicio 11}
\addcontentsline{toc}{section}{Ejercicio 11}
\textbf{¿Sería adecuado usar como descriptor de un punto Harris los valores de los píxeles de su región de soporte? Identifique ventajas, inconvenientes y
mecanismos de superación de estos últimos.}


\section*{Ejercicio 12}
\addcontentsline{toc}{section}{Ejercicio 12}
\textbf{Describa un par de criterios que sirvan para seleccionar parejas de puntos en correspondencias (“matching”) a partir de descriptores de regiones
extraídos de dos imágenes. ¿Por qué no es posible garantizar que todas las parejas son correctas?}


\section*{Ejercicio 13}
\addcontentsline{toc}{section}{Ejercicio 13}
\textbf{Cual es el objetivo principal del uso de la técnica RANSAC en el cálculo de una homografía. Justificar la respuesta}


\section*{Ejercicio 14}
\addcontentsline{toc}{section}{Ejercicio 14}
\textbf{Si tengo 4 imágenes de una escena de manera que se solapan la 1-2, 2-3 y 3-4. ¿Cuál es el número mínimo de parejas de puntos en correspondencias
necesarios para montar un mosaico? Justificar la respuesta}


\section*{Ejercicio 15}
\addcontentsline{toc}{section}{Ejercicio 15}
\textbf{¿En la confección de un mosaico con proyección rectangular es esperable que aparezcan deformaciones geométricas de la escena real? ¿Cuáles y por qué?
¿Bajo qué condiciones esas deformaciones podrían no estar presentes? Justificar la respuesta.}

\end{document}